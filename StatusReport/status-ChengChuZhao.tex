%%%%%%%%%%%%%%%%%%%%%%%%%%%%%%%%%%%%%%%%%%%%%%%%%%%%%%%%%%%%%%%%%%
%%%%%%%% ICML 2014 EXAMPLE LATEX SUBMISSION FILE %%%%%%%%%%%%%%%%%
%%%%%%%%%%%%%%%%%%%%%%%%%%%%%%%%%%%%%%%%%%%%%%%%%%%%%%%%%%%%%%%%%%

% Use the following line _only_ if you're still using LaTeX 2.09.
%\documentstyle[icml2014,epsf,natbib]{article}
% If you rely on Latex2e packages, like most moden people use this:
\documentclass{article}

% use Times
\usepackage{times}
% For figures
\usepackage{graphicx} % more modern
%\usepackage{epsfig} % less modern
\usepackage{subfigure} 

% For citations
\usepackage{natbib}

% For algorithms
\usepackage{algorithm}
\usepackage{algorithmic}

% As of 2011, we use the hyperref package to produce hyperlinks in the
% resulting PDF.  If this breaks your system, please commend out the
% following usepackage line and replace \usepackage{icml2014} with
% \usepackage[nohyperref]{icml2014} above.
\usepackage{hyperref}

% Packages hyperref and algorithmic misbehave sometimes.  We can fix
% this with the following command.
\newcommand{\theHalgorithm}{\arabic{algorithm}}

% Employ the following version of the ``usepackage'' statement for
% submitting the draft version of the paper for review.  This will set
% the note in the first column to ``Under review.  Do not distribute.''
\usepackage{icml2014} 


% The \icmltitle you define below is probably too long as a header.
% Therefore, a short form for the running title is supplied here:
\icmltitlerunning{CHENG, CHU, ZHAO}

\begin{document} 

\twocolumn[
\icmltitle{Objects Recognization Based on On-line Machine Learning\\Project Status Report for CIS 519}

% It is OKAY to include author information, even for blind
% submissions: the style file will automatically remove it for you
% unless you've provided the [accepted] option to the icml2014
% package.
\icmlauthor{Shangyi Cheng}{shangyi@seas.upenn.edu}
\icmlauthor{Yao Chu}{chuyao@seas.upenn.edu}
\icmlauthor{Chenyang Zhao}{chzhao@seas.upenn.edu}

% You may provide any keywords that you 
% find helpful for describing your paper; these are used to populate 
% the "keywords" metadata in the PDF but will not be shown in the document
\icmlkeywords{machine learning, circle detection, feature extraction, objects recognization, neural network}

\vskip 0.3in
]

\begin{abstract} 
To recognize objects (various kinds of ball in our project) in the given images, the model we proposed in this report will take three steps. The first step is using Hough tansform to detect the most possible region that contains the ball. The second step contains techniques to extract feature from the selected region. In the third step, neural network is applied on the extracted features to classify the given instances. Several training/testing experiments have be carried out to show the performance of this model.
\end{abstract}


\section{Introduction}

This work presents an idea of objects recognition. We decided to use images of four kinds of balls from Caltech 256(Griffin, G. Holub, AD. Perona, P.) as our dataset, which contains 98 images of golf in the folder ''088.golf-ball'', 174 in ''193.soccer-ball'', 104 in ''017.bowling-ball'' and 98 in ''216.tennis-ball''.\\
The task is challenging in the following aspects:\\

1. The objects with the same label can have very different colors, sizes and textures.\\
2. The objects we want to classify may not be the main parts in the image, sometimes with even other spherical objects
such as a bowling-ball with a head in ''017.bowling-ball$ \verb|\|$017\_0006.jpg''.\\3. \texttt{T}he objects may not have high contrast against the surroundings, such as a white soccer-ball in a bright background in ''193.soccer-ball $ \verb|\|$193\_0070.jpg''.\\
4. It's possible that several objects with the same label are in one image (such as ''193.soccer-ball $ \verb|\|$ 193\_0171.jpg'').\\
5. Some images are even hard for humans to classify, such as ''017.bowling-ball $ \verb|\|$ 193\_0155.jpg'', which has the color of a typical tennis ball but the pattern of pentagon and hexagon on a soccer ball.\\


\section{Model Details} 


\subsection{Overview of the model}
The model for ball recognization is shown in Figure.1. Ball recognization can be considered as the problem of detect the circular region in the image, then fetch features inside the circle as input features for next step. In the final step, use neural network to classify the processed data into four catagories.\\

\subsection{Circle Detection}
At the beginning, we considered using corner detection directly on the original image to extract features. But we found is not good enough. From Figure.2, we can see that the detected corner features are distributed on the entire image, even more dense in the surroundings when the background is complex. Thus, we think it's necessary to first determine the circular region in each image.
After reading several papers on circle detection, we found there is widely-used method called Circular Hough Transform (CHT) to detect circles, such as goal detection in soccer matches. This approach is used because of its robustness in the presence of noise, occlusion and varying illumination. And MATLAB has an build-in function ''[centers,radii] = imfindcircles(A, radiusRange)''. Given a truecolor image and range of radii for the circular object to detect, it will return the coordinates of circle centers and corresponding estimated radii.\\
There are two other arguments that we find useful to improve the performance of circle detection. One is ''ObjectPolarity'', which can be set to ''bright'' (the circular objects are brighter than the background) or ''dark'' (the circular objects are darker than the background). The other is ''Sensitivity'', which is a scalar value in the range [0,1]. By increasing the sensitivity factor, imfindcircles detects more circular objects, including weak and partially obscured circles but it also increases the risk of false detection.\\
Algorithm~\ref{alg:example} shows the progress of using CHT to detect circles.\\
The main idea is to increase the sensitivity gradually if the there is no detected circle with the lower one. For a certain sensitivity, use several radius ranges combined with bright/dark ObjectPolarity.

\subsection{Results}
 
Several results of applying our circle detection algorithm on Caltech256 dataset are shown in Figure.3.\\
The statics of the performance on four catagories is summarized in Table.1. Note: -1 means the detected circles are all wrong; 0 means it fails to detect any circle; 1 means only circular region that contains the ball is detected; 2 means result contains other false circles; 

\section{Work for Next Stage} 

\subsection{Figures}
 
You may want to include figures in the paper to help readers visualize
your approach and your results. Such artwork should be centered,
legible, and separated from the text. Lines should be dark and at
least 0.5~points thick for purposes of reproduction, and text should
not appear on a gray background.

Label all distinct components of each figure. If the figure takes the
form of a graph, then give a name for each axis and include a legend
that briefly describes each curve. Do not include a title inside the
figure; instead, be sure to include a caption describing your figure.

You may float figures to the top or
bottom of a column, and you may set wide figures across both columns
(use the environment {\tt figure*} in \LaTeX), but always place
two-column figures at the top or bottom of the page. 

\subsection{Algorithms}

If you are using \LaTeX, please use the ``algorithm'' and ``algorithmic'' 
environments to format pseudocode. These require 
the corresponding stylefiles, algorithm.sty and 
algorithmic.sty, which are supplied with this package. 
Algorithm~\ref{alg:example} shows an example. 

\begin{algorithm}[tb]
   \caption{Bubble Sort}
   \label{alg:example}
\begin{algorithmic}
   \STATE {\bfseries Input:} data $x_i$, size $m$
   \REPEAT
   \STATE Initialize $noChange = true$.
   \FOR{$i=1$ {\bfseries to} $m-1$}
   \IF{$x_i > x_{i+1}$} 
   \STATE Swap $x_i$ and $x_{i+1}$
   \STATE $noChange = false$
   \ENDIF
   \ENDFOR
   \UNTIL{$noChange$ is $true$}
\end{algorithmic}
\end{algorithm}
 
\subsection{Tables} 
 
You may also want to include tables that summarize material. Like 
figures, these should be centered, legible, and numbered consecutively. 
However, place the title {\it above\/} the table, as in 
Table~\ref{sample-table}.
% Note use of \abovespace and \belowspace to get reasonable spacing 
% above and below tabular lines. 

\begin{table}[t]
\caption{Classification accuracies for naive Bayes and flexible 
Bayes on various data sets.}
\label{sample-table}
\vskip 0.15in
\begin{center}
\begin{small}
\begin{sc}
\begin{tabular}{lcccr}
\hline
\abovespace\belowspace
Data set & Naive & Flexible & Better? \\
\hline
\abovespace
Breast    & 95.9$\pm$ 0.2& 96.7$\pm$ 0.2& $\surd$ \\
Cleveland & 83.3$\pm$ 0.6& 80.0$\pm$ 0.6& $\times$\\
Glass2    & 61.9$\pm$ 1.4& 83.8$\pm$ 0.7& $\surd$ \\
Credit    & 74.8$\pm$ 0.5& 78.3$\pm$ 0.6&         \\
Horse     & 73.3$\pm$ 0.9& 69.7$\pm$ 1.0& $\times$\\
Meta      & 67.1$\pm$ 0.6& 76.5$\pm$ 0.5& $\surd$ \\
Pima      & 75.1$\pm$ 0.6& 73.9$\pm$ 0.5&         \\
\belowspace
Vehicle   & 44.9$\pm$ 0.6& 61.5$\pm$ 0.4& $\surd$ \\
\hline
\end{tabular}
\end{sc}
\end{small}
\end{center}
\vskip -0.1in
\end{table}

Tables contain textual material that can be typeset, as contrasted 
with figures, which contain graphical material that must be drawn. 
Specify the contents of each row and column in the table's topmost
row. Again, you may float tables to a column's top or bottom, and set
wide tables across both columns, but place two-column tables at the
top or bottom of the page.
 
\subsection{Citations and References} 

Please use APA reference format regardless of your formatter
or word processor. If you rely on the \LaTeX\/ bibliographic 
facility, use {\tt natbib.sty} and {\tt icml2014.bst} 
included in the style-file package to obtain this format.

Citations within the text should include the authors' last names and
year. If the authors' names are included in the sentence, place only
the year in parentheses, for example when referencing Arthur Samuel's
pioneering work \yrcite{Samuel59}. Otherwise place the entire
reference in parentheses with the authors and year separated by a
comma \cite{Samuel59}. List multiple references separated by
semicolons \cite{kearns89,Samuel59,mitchell80}. Use the `et~al.'
construct only for citations with three or more authors or after
listing all authors to a publication in an earlier reference \cite{MachineLearningI}.

The references at the end of this document give examples for journal
articles \cite{Samuel59}, conference publications \cite{langley00}, book chapters \cite{Newell81}, books \cite{DudaHart2nd}, edited volumes \cite{MachineLearningI}, 
technical reports \cite{mitchell80}, and dissertations \cite{kearns89}. 


 
\section*{Acknowledgments} 
  None.

\bibliography{example_paper}
\bibliographystyle{icml2014}

\end{document} 


