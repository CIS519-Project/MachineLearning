%%%%%%%%%%%%%%%%%%%%%%%%%%%%%%%%%%%%%%%%%%%%%%%%%%%%%%%%%%%%%%%%%%
%%%%%%%% ICML 2014 EXAMPLE LATEX SUBMISSION FILE %%%%%%%%%%%%%%%%%
%%%%%%%%%%%%%%%%%%%%%%%%%%%%%%%%%%%%%%%%%%%%%%%%%%%%%%%%%%%%%%%%%%

% Use the following line _only_ if you're still using LaTeX 2.09.
%\documentstyle[icml2014,epsf,natbib]{article}
% If you rely on Latex2e packages, like most moden people use this:
\documentclass{article}

% use Times
\usepackage{times}
% For figures
\usepackage{graphicx} % more modern
%\usepackage{epsfig} % less modern
\usepackage{subfigure} 

% For citations
\usepackage{natbib}

% For algorithms
\usepackage{algorithm}
\usepackage{algorithmic}

% As of 2011, we use the hyperref package to produce hyperlinks in the
% resulting PDF.  If this breaks your system, please commend out the
% following usepackage line and replace \usepackage{icml2014} with
% \usepackage[nohyperref]{icml2014} above.
\usepackage{hyperref}

% Packages hyperref and algorithmic misbehave sometimes.  We can fix
% this with the following command.
\newcommand{\theHalgorithm}{\arabic{algorithm}}

% Employ the following version of the ``usepackage'' statement for
% submitting the draft version of the paper for review.  This will set
% the note in the first column to ``Under review.  Do not distribute.''
\usepackage{icml2014} 


% The \icmltitle you define below is probably too long as a header.
% Therefore, a short form for the running title is supplied here:
\icmltitlerunning{PUT YOUR LAST NAMES HERE}

\begin{document} 

\twocolumn[
\icmltitle{Project Report Template for CIS 419/519\\Introduction to Machine Learning}

% It is OKAY to include author information, even for blind
% submissions: the style file will automatically remove it for you
% unless you've provided the [accepted] option to the icml2014
% package.
\icmlauthor{Your Name}{email@yourdomain.edu}
\icmlauthor{Your CoAuthor's Name}{email@coauthordomain.edu}
\icmlauthor{Your CoAuthor's Name}{email@coauthordomain.edu}

% You may provide any keywords that you 
% find helpful for describing your paper; these are used to populate 
% the "keywords" metadata in the PDF but will not be shown in the document
\icmlkeywords{boring formatting information, machine learning}

\vskip 0.3in
]




\section{Status Report}
\subsection{Circle Hough Transform}
The first step of ball detection and classification is to detect the circle area, where the ball possibly locates. During building the training set, the area can be located by either circle detection algorithm or manually selection. However, when classifying a ball in a new image, the circle area has to be detected automatically. Several methods for circle detection are studied in the last few years,\cite{cirDetect1},\cite{cirDetect2},\cite{cirDetect3}. One of the robustest approach is Circle Hough Transform(CHT), which takes the image $M$ and desired radius range $[r_{min},r_{max}]$ as inputs and return the positions of circle centers. The algorithm could be presented in Algorithm \ref{alg:CHT}. In our case, we use the log phase for the accumulator which is\\
\begin{equation}
\phi_{m,n}^{logr} = 2\pi\Big(\frac{\log{[\sqrt{(m-i)^2+(n-j)^2)}]}-\log{r_{min}}}{\log{r_{max}}-\log{r_{min}}}\Big)
\end{equation}
\begin{algorithm}[tb]
   \caption{Circle Hough Transform}
   \label{alg:CHT}
\begin{algorithmic}
   \STATE {\bfseries Input:} image $I$, radius range $[r_{min},r_{max}]$
   \STATE Initialize accumulator matrix $M = 0$.
   \FOR{each edge point$(i,j)$ in image $I$}
   \FOR{each point $(m,n)$ in image $I$}
   \IF{$r_{min}^2<(m-i)^2+(n-j)^2<r_{max}^2$}
   \STATE {$M(m,n) = M(m,n)+\phi_{m,n}$}
   \ENDIF
   \ENDFOR
   \ENDFOR
   \STATE {\bfseries Return:} positions of local maximum points in accumulator matrix $M$
\end{algorithmic}
\end{algorithm}\\



\bibliography{example_paper}
\bibliographystyle{icml2014}
\
\end{document} 


% This document was modified from the file originally made available by
% Pat Langley and Andrea Danyluk for ICML-2K. This version was
% created by Lise Getoor and Tobias Scheffer, it was slightly modified  
% from the 2010 version by Thorsten Joachims & Johannes Fuernkranz, 
% slightly modified from the 2009 version by Kiri Wagstaff and 
% Sam Roweis's 2008 version, which is slightly modified from 
% Prasad Tadepalli's 2007 version which is a lightly 
% changed version of the previous year's version by Andrew Moore, 
% which was in turn edited from those of Kristian Kersting and 
% Codrina Lauth. Alex Smola contributed to the algorithmic style files.  
